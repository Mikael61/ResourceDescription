\documentclass[a4,12pt]{article}
\usepackage[cp1252]{inputenc}
\usepackage[T1]{fontenc}
\usepackage{times}
\usepackage[swedish]{babel}
%\usepackage[pdftex]{graphicx}
\usepackage{graphicx}
\usepackage{pstricks}
\usepackage{pst-node}
\raggedbottom
% \usepackage{pst-eps}
\usepackage{graphics} 
\usepackage{color} 
%\usepackage{setspace}
%\setstretch{1.3}
\parindent=5mm 
\parskip=0mm
\usepackage[textwidth=127mm,textheight=200mm]{geometry}
%\usepackage{semcolor}
%\usepackage{alltt}
%\usepackage{moreverb}
%\usepackage{parsetree}
%\usepackage{fancyhdr}
\setlength{\unitlength}{1mm}
%\slideframe{none}
%\slidesmag{2}
%\renewcommand{\slideleftmargin}{5mm}
%\renewcommand{\sliderightmargin}{5mm}
%\paperwidth=240mm
%\slidewidth=200mm
%\slideheight=210mm
\begin{document}
\pagestyle{empty}

\Huge
\begin{center}
  Lösningsförslag \\ 


\rule{100mm}{.5mm}

\end{center}
\normalsize



\section{Exempel 1}
\label{sec:ex1}




\vspace{3mm}

\textbf{Attribut till manifestationen}

\noindent $m_{1.1.1}$

 \texttt{Identifikator (ISBN)} 8203161960 ;

 \texttt{Huvudtitel} Felttoget ;

\texttt{Övrig titelinformation} general Otto Ruges erindringer fra kampene april-juni
1940 ; 

\texttt{Upphovsuppgift} redigert og med innledning av Olav Riste ;

\texttt{Upplagebeteckning} 2. opplag ;

 \texttt{Utgivningsort} [Oslo ?] ;

 \texttt{Utgivarnamn} H. Aschehoug \& Co. ;

 \texttt{Utgivningstid} 1989 ;
 
  \texttt{Mediatyp}  omedierad;
 
 \texttt{Bärartyp}  volym;

\texttt{Bäraromfång} 215 sidor ;

\texttt{Övriga fysiska detaljer} illustrationer .



\vspace{3mm}

\textbf{Attribut för personerna}

\noindent ${p_11}$

   \texttt{Föredraget namn för person} Ruge, Otto ;

   \texttt{Levnadstid} 1882-1961 .

\noindent ${p_12}$

   \texttt{Föredraget namn för person} Riste, Olav ;

   \texttt{Levnadstid} 1933- .
\vspace{3mm}


\textbf{Relationer}

\noindent $m_{1.1.1}$

\texttt{Huvudupphov} $p_{11}$ ;

\texttt{Medupphov} [\texttt{id} $p_{12}$ ; \texttt{roll} oth ] .

\vspace{3mm}

\small 

\noindent \texttt{700} [\texttt{a} $p_{2}$ ; \texttt{4} oth ] kan
tyckas konstigt, och är lite grann av en ad-hoc-lösning. Relationen
700 uttrycker ansvar för innehållet, \texttt{a} definierar den entitet
som relateras till manifestationen, medan \texttt{4} kvalifierar
(preciserar) relationens karaktär. Här används koden 'oth' för
``medarbetare (ospecificerad)'', en slags slaskkod alltså. 'oth'
tillhör alltså relationen mellan två entiteter och inte någon av
entiteterna. Man kan säga att det är ett attribut för en relation,
även om det sättet att se på saken inte uttrycks i RDA. Observera att
RDA egentligen förespråkar att 'oth' skall uttryckas i klartext, men
att vi här gör bruk av de treställiga koderna som listas på 
\begin{footnotesize}
  http://www.kb.se/dokument/Verktygsladan/funktionskoder\_full\_funktion.pdf
\end{footnotesize}
med svenska benämningar.

\normalsize

\newpage

\section{Exempel 2}
\label{sec:ex2}



\textbf{Attribut för manifestationen}

\noindent $m_{2.1.1}$

 \texttt{Identifikator (ISBN)} 0817636366 (USA) ;

 \texttt{Identifikator (ISBN)} 3764336366 (Schweiz) ;

 \texttt{Huvudtitel} Discrete thoughts ;
 
\texttt{Övrig titelinformation} essays on mathematics, science and
philosophy ;

\texttt{Upphovsuppgift} Mark Kac, Gian-Carlo Rota, and Jacob T. Schwartz  ;

\texttt{Upplagebeteckning} Revised and corrected edition ;

\texttt{Upplagespecifik upphovsuppgift} with the assistance of Peter Renz ;

 \texttt{Utgivningsort} Boston ;

 \texttt{Utgivningsort} Basel ;

 \texttt{Utgivarnamn} Birkhäuser ;

 \texttt{Utgivningstid} [1992] ;
 
  \texttt{Mediatyp}  omedierad;
 
 \texttt{Bärartyp}  volym;

\texttt{Bäraromfång} xii, 264 sidor ;

\texttt{Övriga fysiska detaljer} illustrationer .

\vspace{3mm}
För att skilja de båda identiska utgåvorna åt (som alltså behandlas
som en manifestation) anges inom parentes till vilken utgåva
respektive ISBN-nr hör. ''If the resource bears more than one
identifier of the same type, record a brief qualification after the
identifier.'' (RDA 2.15.17)

''A statement of responsibility relating to the edition is a statement
relating to the identification of any persons, families, or corporate
bodies responsible for the edition being described but not to all
editions.'' (RDA 2.5.4.1) Skriv av exakt som det anges och notera att
upphovsuppgifter relaterade till utgåvor noteras med \texttt{Upplagespecifik upphovsuppgift 250b}.

\vspace{3mm}

\textbf{Attribut för personerna}

\noindent $p_21$ 

\texttt{Föredraget namn för person} Kac, Mark ; 

\texttt{Levnadstid} 1914-1984 .

\noindent $p_22$

\texttt{Föredraget namn för person} Rota, Gian-Carlo ; 

\texttt{Levnadstid} 1932-1999 .
 


\noindent $p_23$

\texttt{Föredraget namn för person} Schwartz, Jacob T. ; 

\texttt{Levnadstid} 1930-2009 .
 


\noindent $p_24$ 

\texttt{Föredraget namn för person} Renz, Peter ;

\texttt{Levnadstid} 1946- .






\vspace{3mm}

\textbf{Relationer}

\noindent $m_{2.1.1}$

\texttt{Medupphov} [ \texttt{id} $p_21$ ; \texttt{roll} aut ] ;

\texttt{Medupphov} [ \texttt{id} $p_22$ ; \texttt{roll} aut ] ;

\texttt{Medupphov} [ \texttt{id} $p_23$ ; \texttt{roll} aut ] ;

\texttt{Medupphov} [ \texttt{id} $p_24$ ; \texttt{roll} edt ] .



\vspace{3mm}


\noindent Om det inte via LIBRIS är möjligt att fastställa attribut för personer
så behöver du inte gå längre än så.




\newpage


\section{Exempel 3}
\label{sec:ex3}



\textbf{Attribut för manifestationen}

\noindent $m_{3.1.1}$

 \texttt{Identifikator (ISBN)} 9129480973 ;

 \texttt{Huvudtitel} Att förstå när barnen talar ;
 
\texttt{Övrig titelinformation} exempel och analys ;

\texttt{Upphovsuppgift} Nancy Martin, Paul Williams, Joan Wilding, Susan Hemmings och Peter Medway ; med förord av Lars Jalmert ;
översättning av Synnöve Olsson ;

 \texttt{Utgivningsort} [Stockholm] ;

 \texttt{Utgivarnamn} Rabén \& Sjögren ;

 \texttt{Utgivningstid} 1977 ;
 
  \texttt{Mediatyp}  omedierad;
 
 \texttt{Bärartyp}  volym;

\texttt{Bäraromfång} 212 sidor ;

\texttt{Series huvudtitel} Tema nova ;

\texttt{Allmänna anmärkningar} Originalets titel: Understanding children talking .

\vspace{3mm}
Den här typen av kommentater som originaltiteln utgör kan noteras med
Allmänna anmärkningar (500a), men är inte obligatoriskt. Särskilt eftersom en relation till
uttrycket också skall anges. 
\vspace{3mm}

\textbf{Attribut för relaterade verk/uttryck/manifestationer}

\noindent $m_{3.2.1}$

  \texttt{Huvudtitel} Understanding children talking .

\noindent $m_{30.1.1}$

  \texttt{Huvudtitel} Tema nova .



\vspace{3mm}

\textbf{Attribut för personerna}

\noindent $p_{31}$

  \texttt{Föredraget namn för person} Martin, Nancy .

\noindent $p_{32}$

  \texttt{Föredraget namn för person} Medway, Peter .

\noindent $p_{33}$

  \texttt{Föredraget namn för person} Williams, Paul .

\noindent $p_{34}$

  \texttt{Föredraget namn för person} Wilding, Joan .

\noindent $p_{35}$

  \texttt{Föredraget namn för person} Hemmings, Susan .
  
  \noindent $p_{36}$

  \texttt{Föredraget namn för person} Olsson, Synnöve .



\vspace{3mm}

\textbf{Relationer}

\noindent $m_{3.1.1}$

\texttt{Huvudupphov} [ \texttt{id}  $p_{31}$ ; \texttt{roll}  aut ] ;

\texttt{Medupphov} [ \texttt{id}  $p_{32}$ ; \texttt{roll}  aut ] ;

\texttt{Medupphov} [ \texttt{id}  $p_{33}$ ; \texttt{roll}  aut ] ;

\texttt{Medupphov} [ \texttt{id}  $p_{34}$ ; \texttt{roll}  aut ] ;

\texttt{Medupphov} [ \texttt{id}  $p_{35}$ ; \texttt{roll}  aut ] ;

\texttt{Medupphov} [ \texttt{id}  $p_{36}$ ; \texttt{roll}  trl ] ;

\texttt{Serie}  $m_{30.1.1}$ ;

\texttt{Uniform titel} $m_{3.2.1}$ .

\vspace{3mm}

\noindent 

Förordsförfattaren betraktas som medupphov när förordet är omfattande (25%) enligt nuvarande praxis.






\newpage

\section{Exempel 4}
\label{sec:ex4}


\textbf{Attribut för manifestationen}

\noindent $m_{4.1.1}$

 \texttt{Identifikator (ISBN)} 0949060887 ;

 \texttt{Huvudtitel} Information literacy around the world ;
 
\texttt{Övrig titelinformation} advances in
programs and research ;

\texttt{Upphovsuppgift} edited by Christine Bruce and Philip
Candy ; editorial assistance Helmut Klaus ;

 \texttt{Utgivningsort} Wagga Wagga, New South Wales ;

 \texttt{Utgivarnamn} Centre for Information Studies ;

 \texttt{Utgivningstid} 2000 ;
 
  \texttt{Mediatyp}  omedierad;
 
 \texttt{Bärartyp}  volym;

\texttt{Bäraromfång} xvi, 304 sidor ;

\texttt{Serie} [ \texttt{huvudtitel} Occasional publications / Centre for Information Studies,
Charles Sturt University ; \texttt{volym} 1 ] .

\vspace{3mm}

RDA 2.12.6.3 har ''Record statements of responsibility associated with
the series title only if they are considered to be necessary for
identification of the series.'' och det måste anses att serietiteln i
sig inte är särskilt signifikant och sannolikt kan sammanblandas med
annan serie, varför upphovet blir viktigt. Notera ISBD-interpunktionen
'/' som skiljer titeln från upphovsuppgiften. 


\vspace{3mm}



\textbf{Attribut för relaterad manifestation}

\noindent $m_{5.1.1}$

  \texttt{Huvudtitel} Occasional publications ;

  \texttt{Upphovsuppgift} Centre for Information Studies,
Charles Sturt University ;

\texttt{Identifikator (ISSN)} 1443-4334 .




\vspace{3mm}

\textbf{Attribut för personerna}

\noindent $p_{41}$
  \texttt{Föredraget namn för person} Bruce, Christine .

\noindent $p_{42}$
  \texttt{Föredraget namn för person} Klaus, Helmut .

\noindent $p_{43}$
  \texttt{Föredraget namn för person} Candy, Philip .


\vspace{3mm}

\textbf{Relationer}

\noindent $m_{4.1.1}$

\texttt{Medupphov} [ \texttt{id}  $p_{41}$ ; \texttt{roll}  edt ] ;

\texttt{Medupphov} [ \texttt{id}  $p_{42}$ ; \texttt{roll}  oth ] ;

\texttt{Medupphov} [ \texttt{id}  $p_{43}$ ; \texttt{roll}  edt ] ;

\texttt{Serie}  $m_{5.1.1}$ .





\newpage

\section{Exempel 5}
\label{sec:ex5}


\textbf{Attribut för manifestationen}

\noindent $m_{6.1.1}$

 \texttt{Identifikator (ISBN)} 9189140443 ;

 \texttt{Identifikator (ISBN)} 9789189140448 ;

 \texttt{Huvudtitel} Bortom etnicitet ;
 
\texttt{Övrig titelinformation} festskrift till Aleksandra Ålund ;

\texttt{Upphovsuppgift} Diana
Mulinari \& Nora Räthzel, red. ; författare: Ursula Apitzsch [och tjugo andra] ;


\texttt{Upplagebeteckning} Första upplagan ;

 \texttt{Utgivningsort} Umeå ;

 \texttt{Utgivarnamn} Boréa ;

 \texttt{Utgivningstid} 2006 ;
 
  \texttt{Mediatyp}  omedierad;
 
 \texttt{Bärartyp}  volym;

\texttt{Bäraromfång} 250 sidor .

\vspace{3mm}


\textbf{Attribut för personerna}

\noindent $p_{51}$
 
 \texttt{Föredraget namn för person} Ålund, Aleksandra ;

  \texttt{Levnadstid} 1945- .

\noindent $p_{52}$

  \texttt{Föredraget namn för person} Mulinari, Diana ;

  \texttt{Levnadstid} 1954- .

\noindent $p_{53}$

  \texttt{Föredraget namn för person} Räthzel, Nora .

\noindent $p_{54}$

  \texttt{Föredraget namn för person} Apitzsch, Ursula .

  RDA 2.4.1.5 har en möjlighet till arbetsbesparing som det är upp
  till varje instans att ta ställning till om den skall tillämpa ''If
  a single statement of responsibility names more than three persons,
  families, or corporate bodies performing the same function, or with
  the same degree of responsibility, omit all but the first of each
  group of such persons, families, or bodies. Indicate the omission by
  summarizing what has been omitted in the language and script
  preferred by the agency preparing the description. Indicate that the
  summary was taken from a source outside the resource itself as
  instructed under 2.2.4''

  Här handlar det om 20 personer, varför vi nog kan säga att
  undantaget sannolikt kan komma att tillämpas, och vi anger endast
  den först nämnda. Enligt nuvarande praxis skulle fält 100 saknas,
  och vi gör så i det här fallet, eftersom digniteten hos vars och ens
  bidrag är svår att avgöra. I strikt mening är det betydelselöst om
  vi har med 100 eller inte, som understrukits på annan plats
  här. Föremålet för festskriften är också associerad med
  manifestationen/verket och skall redovisas.

\vspace{3mm}

\textbf{Relationer}

\noindent $m_{6.1.1}$

\texttt{Medupphov} [ \texttt{id}  $p_{51}$ ; \texttt{roll}  hnr ] ;

\texttt{Medupphov} [ \texttt{id}  $p_{52}$ ; \texttt{roll}  edt ] ;

\texttt{Medupphov} [ \texttt{id}  $p_{53}$ ; \texttt{roll}  edt ] ;

\texttt{Medupphov} [ \texttt{id}  $p_{54}$ ; \texttt{roll}  aut ] .







\end{document}


